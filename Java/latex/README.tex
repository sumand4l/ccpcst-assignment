\documentclass{report}
\usepackage[utf8]{inputenc}

% **********************************************************
% Document Settings                                        *
% **********************************************************
\usepackage{geometry}
\geometry{a4paper, margin=0.6in}

\usepackage{graphicx}
\graphicspath{{../outputs/}}

\usepackage{minted}
% \usemintedstyle{emacs}

\usepackage{fancyhdr}
\pagestyle{fancy}
\fancyhf{}
\lhead{Java Programming}
\rhead{Suman Mondal}
% \lfoot{\href{https://github.com/thatsuman/ccpcst-assignment.git}{View on Github}}
\lfoot{\href{https://github.com/thatsuman}{github.com/thatsuman}}
\rfoot{Page \thepage}

\usepackage{hyperref}
\hypersetup{
colorlinks=true,
linkcolor=blue,
filecolor=magenta,
urlcolor=blue,
}
\urlstyle{same}

\usepackage{fontspec}
% \setromanfont{Source Sans Pro}
% \setmonofont{Cascadia Code PL}


% macros start here
\newcommand{\problem}[3]{
  \section{#3}
    \underline{{\LARGE Source Code :}}
    \inputminted[breaklines,fontsize=\Large]{java}{#1.java}
    \bigbreak
    \noindent
    \underline{{\LARGE Program Output :}}
    \bigbreak
    \noindent
    \includegraphics[height=2.1in]{#2}
% \newpage
}
\newcommand{\subproblem}[3]{
  \subsection{#3}
    \underline{\emph{\Large Source Code :}}
    \inputminted[breaklines]{java}{#1.java}
    \bigbreak
    \noindent
    \underline{\emph{\Large Program Output :}}
    \bigbreak
    \noindent
    \includegraphics{#2}
% \newpage
}
% macros end here

% **********************************************************

\begin{document}
\begin{titlepage}
  \begin{center}
    \vspace*{2cm}
    \includegraphics[width=0.3\textwidth]{./outputs/logo}\\
    \vspace{0.5cm}
    {\huge \textbf{CENTRAL CALCUTTA POLYTECHNIC}}\\
    \vspace{0.4cm}
    21, Convent Road, Philips, Sealdah, Kolkata, West Bengal 700014\\
    \vspace{0.8cm}
    {\Large \textsc{dept. : computer science and technology}}
  \end{center}
  \vspace{1.2cm}
  \textsc{
    \huge
    \begin{itemize}
      \item name : suman mondal
      \item roll : dccpcsts5
      \item number : 10005537
      \item reg number : d192005242
      \item subject : multimedia and animation technique
      \item session : 2021 - 2022
      \item email : suman.mondal@outlook.in
    \end{itemize}
  }

\end{titlepage}

\pagenumbering{roman}
\newpage
\large{\tableofcontents}
% \tableofcontents
\clearpage
\pagenumbering{arabic}

\chapter{Java Assignment}
\problem{../codes/Q01}{Q01}{Print the average of three numbers entered by user by creating a class named ‘Average’ having a method to calculate and print the average}
\problem{../codes/Q02}{Q02}{Print the sum, differences and product of two complex numbers by creating a class named ‘Complex’ with separate methods for each operation whose real and imaginary part is to be entered by the user}
\problem{../codes/Q03}{Q03}{Write a program that would print the information (name, year of joining, salary, address) of three employees by creating a class ‘Employee’. The output should be in a tabular form}
\problem{../codes/Q04}{Q04}{Write a program to input the details of a student using constructor and display the sam}
\problem{../codes/Q05}{Q05}{Write a program to print the information of three employees by creating a class ‘Employee‘ and show the details of all three Employees using Abstract class}
\problem{../codes/Q06}{Q06}{Write a program to give the example for ‘this’ operator. And also use the ‘this’ keyword as a return statement}
\problem{../codes/Q07}{Q07}{Write a program to add all the elements of a One-Dimensional array}
\problem{../codes/Q08}{Q08}{Write a program to reverse the elements of a One-Dimensional array}
\problem{../codes/Q09}{Q09}{Write a program to perform addition, subtraction, multiplication and division of two One-Dimensional arrays}
\problem{../codes/Q10}{Q10}{Write a program to perform addition of two Two-Dimensional arrays}
\problem{../codes/Q11}{Q11}{Write a program to take a string as input and display the string and its length. (Using string Functions)}
\problem{../codes/Q12}{Q12}{Write a program to check whether the inputted string is a Palindrome string or not}
\problem{../codes/Q13}{Q13}{Java Program to count Total number of characters in a string}
\problem{../codes/Q14}{Q14}{Java Program to count the total number of vowels and consonants in a string}
\problem{../codes/Q15}{Q15}{Java Program to remove all the white spaces from a string}
\problem{../codes/Q16}{Q16}{Java program to find the duplicate characters in a string}
\problem{../codes/Q17}{Q17}{Java program to swap two string variables without using third or temp variable}
\problem{../codes/Q18}{Q18}{ Write a program that accepts a shopping list of five items from the command line and stores in a Vector}
\problem{../codes/Q19}{Q19}{Modify the program of Q.18
to accomplish the following:
- To delete an item in
  the list
- To add an item at a
  specified location in
  the list
- To add an item at the
  end of the list
- To print the contents
  of the vector}

\end{document}

